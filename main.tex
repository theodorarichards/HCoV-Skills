\documentclass[a4paper,11pt]{article}

\usepackage[utf8]{inputenc}
\usepackage[UKenglish]{babel}
\usepackage{mathtools,amssymb,amsthm}
\usepackage{adjustbox,graphicx}
\usepackage{hyperref}
\usepackage{lineno}
\include{linenopatch}
\linenumbers % comment out this line if you want to hide the line numbers

\newtheorem{prop}{Proposition} % This defines a theorem-like environment. See https://www.overleaf.com/learn/latex/theorems_and_proofs
\newtheorem{lem}[prop]{Lemma}
\newtheorem{thm}[prop]{Theorem}
\newtheorem{cor}[prop]{Corollary}
\theoremstyle{definition}
\newtheorem{defn}[prop]{Definition}
\newtheorem{ex}[prop]{Example}
% ========= Bibliography =========
% These lines load the `biblatex' package
% and read in the list of references from
% References.bib - take a look.
%
% To generate References.bib, I recommend https://www.mybib.com/
% rather than trying to write the .bib file yourself.
%
\usepackage{csquotes,biblatex}
\addbibresource{References.bib}
% ================================

\title{The Cross Ratio}
\author{Theodora Richards, s1723589}

\begin{document}
\maketitle

\section{Introduction}
The cross ratio gives us a way of identifying mapping between points in the complex plane via a Möbius transformation. It is invariant of projective transformations and the Möbius transformations preserve it.
The cross ratio is defined to be the image of $z_4$ that maps three distinct points, $(z_1, z_2, z_3)$, in the extended complex plane to the three distinct values $1, 0, \infty$ by a Möbius transformation. These three initial points always uniquely define a circle or a line. \\
We have previously been studying Möbius transformations, which preserve the cross ratio. 
The cross ratio is shown by:
$$ [z_1, z_2, z_3, z_4] = \frac{z_1 - z_3}{z_1 - z_4}\frac{z_2 - z_4}{z_2 - z_3}$$

We can permute four points among themselves in $4!$ ways (24) but we actually only get exactly 6 different results. \\
\subsection{Real and Uniqueness}
The cross ratio is real if and only if all the points $(z_1, z_2, z_3, z_4)$ lie on a single circle or Euclidean line. \\
From our lecture notes we get a theorem for the unique Möbius transformation. {\autocite{bickerton_2021_honours}}
\begin{thm}{\autocite{bickerton_2021_honours}}
Let $z_1, z_2, z_3, z_4 \in \Tilde{\mathbb{C}}$ be three distinct points. Then there exists a unique Möbius transformation $f$ such that
$$ f(z_2) = 1, f(z_3) = 0, \text{and} f(z_4) = \infty. $$
\end{thm}
This theorem shows us that any circle or line can be mapped to the real line.
The proof of this theorem can be found in Section 2.5 of the lecture notes.{\autocite{bickerton_2021_honours}} \\
\subsection{Cross ratio and Circles}
Also, in Chapter 13 in Bob Connelly's notes on \textit{``Classic Geometries''}, he gives a very good explanation of how the cross ratio works in conjunction with circles along with a physical representation of what the cross ratio shows us, as shown in Figure 3 where $\theta_1$ and $\theta_3$ are found from the polar decomposition of the complex number $z = re^{i\theta}$, where the magnitude of $z$ is shown by $r = |z|$, and the line through $0$ and $z$ from the real axis gives an angle of $\theta$. This can be seen in Figure 1 below.
\begin{figure}[h]
    \adjincludegraphics[width=0.55\textwidth,center]{cross_ratio1.png} % This uses the adjustbox package to centre the image - the code is slightly neater than the alternatives
    \caption{Figure denoting $(r, z, \theta)$ and the real axis in the complex plane.}
    \label{fig:cross_ratio1}
\end{figure}
If we have the complex numbers $z_1, z_2, z_4$, the polar decomposition of the ratio for the angle $\theta_1$ can be written and thought of as the angle between two vectors $z_2-z_1$ and $z_4-z_1$. This is represented in Figure 2.
\begin{figure}[h]
    \adjincludegraphics[width=0.55\textwidth,center]{cross_ratio2.png} % This uses the adjustbox package to centre the image - the code is slightly neater than the alternatives
    \caption{Figure denoting $(z_1, z_2, z_4)$ and $\theta_1$ in the complex plane.}
    \label{fig:cross_ratio2}
\end{figure}
Finally, we get to our cross ratio $r$ of $z_1, z_2, z_3, z_4$ where $\theta_3$ is the angle at $z_3$ in the quadrilateral determined by these four points. 
\begin{align}
    \phantom{r = (\frac{z_1-z_2}{z_1-z_4})(\frac{z_3-z_4}{z_3-z_2})}
    &\begin{aligned}
        \mathllap{} &= |\frac{z_1-z_2}{z_1-z_4}|e^{i\theta_1}|\frac{z_3-z_4}{z_3-z_2}|e^{i\theta_3}\\
            &\qquad = |\frac{z_1-z_2}{z_1-z_4}||\frac{z_3-z_4}{z_3-z_2}|e^{i(\theta_1+\theta_3)}
    \end{aligned}
\end{align}
Figure 3 represents all of this and what the cross ratio represents.
\begin{figure}[h]
    \adjincludegraphics[width=0.55\textwidth,center]{cross_ratio3.png} % This uses the adjustbox package to centre the image - the code is slightly neater than the alternatives
    \caption{Figure denoting $(z_1, z_2, z_3, z_4)$ in the complex plane.}
    \label{fig:cross_ratio3}
\end{figure}
More detail on this can be found in Chapter 13 of Connelly's notes. {\autocite{Connelly_2017_Math}}\\
\subsection{Mapping}
\begin{ex}
To find a map with Möbius transformations from $(1, i, -1, -i)$ to $(i, -1, -i, 1)$ we put these both into our formula for the cross ratio. If they are equal it means a map exists.
$$ (1, i, -1, -i) = \frac{i + i}{i + 1}\frac{1 + 1}{1 + i} = \frac{4i}{2i} = 2 $$
$$ (i, -1, -i, 1) = \frac{-1-1}{-1+i}\frac{i+i}{i-1} = \frac{-4i}{-2i} = 2. $$
Therefore, we can see these cross ratios are equal which means we can map $(1, i, -1, -i)$ to $(i, -1, -i, 1)$ with a Möbius transformation (rotation). 
\end{ex}
We also get from Bak and Newmans book \textit{``Complex Analysis''} another theorem stating the uniqueness of the bilinear transformation $w$.{\autocite{Bak_Newman_2010_Complex}}
\begin{thm}
The unique bilinear transformation $w = f(z)$ mapping $z_1, z_2, z_3$ to $w_1, w_2, w_3$ is given by:
$$ \frac{(w-w_2)(w_3-w_1)}{(w-w_1)(w_3-w_2)} = \frac{(z-z_2)(z_3-z_1)}{(z-z_1)(z_3-z_2)}. $$
\end{thm}{\autocite{Bak_Newman_2010_Complex}}
Using this new theorem we can show its use with an example.
\begin{ex}
Let $z_1 = 1, z_2 = 2, z_3 = 7$ and let $w_1 = 1, w_2 = 2, w_3 = 3$. Putting these all into our formula, we then solve for $w$ to get:
$$ \frac{(w-2)(3-1)}{(w-1)(3-2)} = \frac{(z-2)(7-1)}{(z-1)(7-2)} $$
$$ w = \frac{7z-4}{2z+1}$${\autocite{Bak_Newman_2010_Complex}}
\end{ex}

%Example 
%$$ (0, 1, i its , -1) = \frac{(0 - i)(1 - (-1))}{(1-i)(-1-0)}$$

%\begin{lem}[{\autocite[Lemma 3.1.2]{bickerton_2021_honours}}]\label{lem:egRes}
%Example result.
%\end{lem}
%\begin{proof}
%Example proof of Lemma~\ref{lem:egRes}.
%\end{proof}

%A reference to a reputable website~\autocite{weisstein_matchstick}. The bib entry was generated automatically using %\url{https://www.mybib.com/}

%An example of a figure is shown in Figure~\ref{fig:knot}.
%\begin{figure}
%    \adjincludegraphics[width=0.25\textwidth,center]{figures/knot.pdf} % This uses the adjustbox package to centre the image - the code is slightly neater than the alternatives
%    \caption{The unofficial logo of the School of Mathematics}
%    \label{fig:knot}
%\end{figure}

\printbibliography % This command prints the cited references.
\end{document}